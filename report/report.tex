\documentclass{article}
\usepackage[main=polish,english]{babel}
\usepackage[utf8]{inputenc}
\usepackage[T1]{fontenc}
\usepackage{cleveref}

\title{Zastosowanie programowania ewolucyjnego w celu rozwiązania problemu
    komiwojażera}

\author{Maciej Wieczór-Retman, Kacper Stojek}

\date{} %leave blank

\begin{document}

\maketitle

\section{Opis problemu}

Problem komiwojażera jest problemem minimalizacji, polegającym na odnalezieniu
minimalnego cyklu Hamiltona w pełnym grafie ważonym. Cykl Hamiltona to taki cykl
w grafie, w którym każdy wierzchołek grafu jest odwiedzony dokładnie raz. Czyli
rozwiązaniem problemu komiwojażera jest cykl o najmniejszej sumie wag krawędzi
grafu. Liczba różnych cyklów Hamiltona $H$ dla pełnego grafu nieskierowanego o
$n$ wierzchołkach jest wyrażana równaniem \cref{equ:n_cycles}.
\begin{equation}
    H = \frac{\left(n - 1\right)!}{2}
    \label{equ:n_cycles}
\end{equation}
Z uwagi na liczbę różnych cyklów, przeszukiwanie wszystkich rozwiązań jest
fizycznie niemożliwe dla większych grafów. Przykładowo stosunkowo mały graf o
stu wierzchołkach, daje w przybliżeniu $4,7 \times  10^{155}$ rozwiązań. Dla
przybliżenia zakładając, że sprawdzenie jednej ścieżki zajmuje $1 \, \mu s$,
przeanalizowanie wszystkich ścieżek zajęłoby $1,48 \times  10^{142} \, lat$.
Z tego powodu ważne jest zastosowanie optymalnych algorytmów minimalizacyjnych.

\section{Implementacja}

\section{Wyniki pracy}

\end{document}